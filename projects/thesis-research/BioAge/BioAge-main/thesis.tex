\documentclass[14pt,a4paper]{report}

% ===== PACKAGES =====
\usepackage[utf8]{inputenc}
\usepackage[T1]{fontenc}
\usepackage{geometry}
\usepackage{setspace}
\usepackage{titlesec}
\usepackage{tocloft}
\usepackage{fancyhdr}
\usepackage{graphicx}
\usepackage{amsmath}
\usepackage{booktabs}
\usepackage{longtable}
\usepackage{hyperref}

% ===== PAGE SETUP (3.3cm margins) =====
\geometry{
    a4paper,
    left=3.3cm,
    right=3.3cm,
    top=3.3cm,
    bottom=3.3cm
}

% ===== FONT SIZE 14pt =====
\renewcommand{\normalsize}{\fontsize{14}{16}\selectfont}

% ===== SPACING =====
\onehalfspacing

% ===== CHAPTER FORMATTING (Each chapter on new page) =====
\titleformat{\chapter}[display]
    {\normalfont\Large\bfseries\centering}
    {Chapter \thechapter}
    {20pt}
    {\LARGE}
\titlespacing*{\chapter}{0pt}{-50pt}{40pt}

% ===== NO SEPARATORS =====
\setlength{\parindent}{0pt}
\setlength{\parskip}{12pt}

% ===== DOCUMENT START =====
\begin{document}

% ========================================
% TITLE PAGE (Page i)
% ========================================
\begin{titlepage}
\centering

\textbf{\large Cairo University}\\[0.3cm]
\textbf{\large Faculty of Graduate Studies for Statistical Research}\\[0.3cm]
\textbf{\large Data Science Program}\\[2cm]

\rule{\linewidth}{0.5mm}\\[0.4cm]
{\LARGE \textbf{Optimizing Actuarial Pricing and Risk Mitigation using Wearable Data and Deep Learning}}\\[0.2cm]
{\Large \textbf{A Comparative Study of Biological Age vs. Chronological Age}}\\[0.1cm]
\rule{\linewidth}{0.5mm}\\[2cm]

\textbf{\large Presented By}\\[0.3cm]
{\Large Ahmed Eltaweel}\\[0.2cm]
\textit{Data Science Program, Faculty of Graduate Studies for Statistical Research}\\[2cm]

\textbf{\large Supervised by}\\[0.5cm]

\begin{minipage}[t]{0.45\textwidth}
\centering
\textbf{Prof. Abdul Hadi Nabih Ahmed}\\
Professor of Applied Statistics,\\
Faculty of Graduate Studies\\
for Statistical Research,\\
Cairo University
\end{minipage}
\hfill
\begin{minipage}[t]{0.45\textwidth}
\centering
\textbf{Prof. Mohammed Reda Abonazel}\\
Professor of Applied Statistics\\
and Econometrics,\\
Faculty of Graduate Studies\\
for Statistical Research,\\
Cairo University
\end{minipage}

\vfill

\textbf{\large A Thesis Submitted to}\\
the Data Science Program\\
in Partial Fulfillment of the Requirements for the\\
\textbf{Degree of Master of Science in Data Science}\\[1cm]

\textbf{\large 2024}

\end{titlepage}

% ========================================
% APPROVAL SHEET (Page ii)
% ========================================
\newpage
\thispagestyle{empty}
\centering

\textbf{\large Cairo University}\\
\textbf{\large Faculty of Graduate Studies for Statistical Research}\\
\textbf{\large Data Science Program}\\[2cm]

\textbf{\Large Approval Sheet}\\[1cm]

\rule{\linewidth}{0.5mm}\\[0.3cm]
{\large \textbf{Optimizing Actuarial Pricing and Risk Mitigation using Wearable Data and Deep Learning}}\\[0.1cm]
\rule{\linewidth}{0.5mm}\\[1cm]

\textbf{Presented By}\\
{\large Ahmed Eltaweel}\\[1.5cm]

A Thesis Submitted to the Data Science Program\\
In Partial Fulfillment of the Requirements for the\\
\textbf{Degree of Master of Science in Data Science}\\[2cm]

\textbf{Approved by the Examining Committee:}\\[1cm]

\begin{tabular}{p{8cm}p{5cm}}
\textbf{Name} & \textbf{Signature} \\[0.5cm]
\hline
Prof. Abdul Hadi Nabih Ahmed & \rule{4cm}{0.4pt} \\[0.8cm]
Prof. Mohammed Reda Abonazel & \rule{4cm}{0.4pt} \\[0.8cm]
Prof. \rule{5cm}{0.4pt} & \rule{4cm}{0.4pt} \\[0.8cm]
Prof. \rule{5cm}{0.4pt} & \rule{4cm}{0.4pt} \\
\end{tabular}

\vfill
\textbf{ii}

% ========================================
% ACKNOWLEDGMENTS (Page iii)
% ========================================
\newpage
\chapter*{Acknowledgments}
\addcontentsline{toc}{chapter}{Acknowledgments}

I would like to express my deepest gratitude to my supervisors, \textbf{Prof. Abdul Hadi Nabih Ahmed} and \textbf{Prof. Mohammed Reda Abonazel}, for their invaluable guidance, continuous support, and insightful feedback throughout this research journey. Their expertise in actuarial science and machine learning has been instrumental in shaping this work.

I am also grateful to the \textbf{Faculty of Graduate Studies for Statistical Research} and the \textbf{Data Science Program} for providing the academic environment and resources necessary to conduct this research.

Special thanks to the \textbf{National Center for Health Statistics (NCHS)} for making the NHANES data publicly available, enabling researchers worldwide to advance the fields of public health and predictive analytics.

Finally, I dedicate this work to my family for their unwavering encouragement and patience.

\vfill
\centering\textbf{iii}

% ========================================
% SUMMARY (Page iv)
% ========================================
\newpage
\chapter*{Summary}
\addcontentsline{toc}{chapter}{Summary}

The importance of developing dynamic actuarial pricing methods has grown significantly due to the limitations of traditional chronological age-based models. This thesis presents the first actuarial application of biological age estimation using NHANES biomarker data for life insurance pricing optimization.

The study employs PhenoAge methodology (Levine et al., 2018) with empirical calibration, achieving validated results on N=4,894 participants from the NHANES 2017-2018 cycle. A Deep Learning Survival Analysis (DeepSurv) framework and Gradient Boosting Survival model (XGBAge) are used to model non-linear interactions between biomarkers and mortality risk.

\textbf{Key Findings:}
\begin{itemize}
    \item \textbf{Gini Coefficient}: 0.332 (50.9\% improvement over chronological age alone)
    \item \textbf{C-Index}: DeepSurv achieved 0.764, outperforming CoxPH (0.687) by 11.2\%
    \item \textbf{Risk Separation}: 45$\times$ separation between healthiest and highest-risk individuals
    \item \textbf{Accelerated Agers}: 13.1\% of population (biological age $>$ chronological by 5+ years)
\end{itemize}

The contribution includes a novel ``MoveDiscount'' dynamic pricing framework that links physical activity to premium adjustments. This research bridges the gap between medical biomarker research and actuarial practice, providing a validated framework for the Egyptian insurance market.

\vfill
\centering\textbf{iv}

% ========================================
% LIST OF ABBREVIATIONS (Page v)
% ========================================
\newpage
\chapter*{List of Abbreviations}
\addcontentsline{toc}{chapter}{List of Abbreviations}

\begin{longtable}{p{3cm}p{10cm}}
\textbf{AI} & Artificial Intelligence \\[0.3cm]
\textbf{ALP} & Alkaline Phosphatase \\[0.3cm]
\textbf{BioAge} & Biological Age \\[0.3cm]
\textbf{C-Index} & Concordance Index \\[0.3cm]
\textbf{CDC} & Centers for Disease Control and Prevention \\[0.3cm]
\textbf{CoxPH} & Cox Proportional Hazards Model \\[0.3cm]
\textbf{CRP} & C-Reactive Protein \\[0.3cm]
\textbf{DeepSurv} & Deep Learning Survival Analysis \\[0.3cm]
\textbf{DL} & Deep Learning \\[0.3cm]
\textbf{EDA} & Exploratory Data Analysis \\[0.3cm]
\textbf{FRA} & Financial Regulatory Authority (Egypt) \\[0.3cm]
\textbf{Gini} & Gini Coefficient \\[0.3cm]
\textbf{HR} & Hazard Ratio \\[0.3cm]
\textbf{IDF} & International Diabetes Federation \\[0.3cm]
\textbf{IoT} & Internet of Things \\[0.3cm]
\textbf{MCV} & Mean Cell Volume \\[0.3cm]
\textbf{MENA} & Middle East and North Africa \\[0.3cm]
\textbf{ML} & Machine Learning \\[0.3cm]
\textbf{NHANES} & National Health and Nutrition Examination Survey \\[0.3cm]
\textbf{PhenoAge} & Phenotypic Age \\[0.3cm]
\textbf{RDW} & Red Cell Distribution Width \\[0.3cm]
\textbf{SD} & Standard Deviation \\[0.3cm]
\textbf{WBC} & White Blood Cell Count \\[0.3cm]
\textbf{WHO} & World Health Organization \\[0.3cm]
\textbf{XGBAge} & XGBoost Survival Age Model \\[0.3cm]
\end{longtable}

\vfill
\centering\textbf{v}

% ========================================
% TABLE OF CONTENTS (Page vi)
% ========================================
\newpage
\tableofcontents
\addcontentsline{toc}{chapter}{Table of Contents}

% ========================================
% LIST OF TABLES (Page vii)
% ========================================
\newpage
\listoftables
\addcontentsline{toc}{chapter}{List of Tables}

% ========================================
% LIST OF FIGURES (Page viii)
% ========================================
\newpage
\listoffigures
\addcontentsline{toc}{chapter}{List of Figures}

% ========================================
% CHAPTER 1: INTRODUCTION
% ========================================
\newpage
\chapter{Introduction}

\begin{center}
\fbox{
\begin{minipage}{0.8\textwidth}
\textbf{1.1 Introduction and Background}

\textbf{1.2 Problem Statement}

\textbf{1.3 Research Objectives}

\textbf{1.4 Research Questions}

\textbf{1.5 Significance of the Study}

\textbf{1.6 Structure of the Thesis}
\end{minipage}
}
\end{center}
\vspace{1cm}

\section{Introduction and Background}
\section{Problem Statement}
\section{Research Objectives}
\section{Research Questions}
\section{Significance of the Study}
\section{Structure of the Thesis}

% ========================================
% CHAPTER 2: LITERATURE REVIEW
% ========================================
\newpage
\chapter{Literature Review}

\begin{center}
\fbox{
\begin{minipage}{0.8\textwidth}
\textbf{2.1 The Evolution of Actuarial Risk Assessment}

\textbf{2.2 Biological Aging Clocks: From DNA to Phenotype}

\textbf{2.3 Wearable Technology in Healthcare and Insurance}

\textbf{2.4 Machine Learning in Survival Analysis}

\textbf{2.5 Recent Advances (2020-2024)}

\textbf{2.6 Research Gap and Contribution}
\end{minipage}
}
\end{center}
\vspace{1cm}

\section{The Evolution of Actuarial Risk Assessment}
\section{Biological Aging Clocks: From DNA to Phenotype}
\section{Wearable Technology in Healthcare and Insurance}
\section{Machine Learning in Survival Analysis}
\section{Recent Advances (2020-2024)}
\section{Research Gap and Contribution}

% ========================================
% CHAPTER 3: RESEARCH METHODOLOGY
% ========================================
\newpage
\chapter{Research Methodology}

\begin{center}
\fbox{
\begin{minipage}{0.8\textwidth}
\textbf{3.1 Research Design}

\textbf{3.2 Data Source: NHANES (2017-2018)}

\textbf{3.3 Data Pre-processing and Feature Engineering}

\textbf{3.4 Biological Age Calculation (PhenoAge)}

\textbf{3.5 Model Development}

\textbf{3.6 Evaluation Methods and Metrics}
\end{minipage}
}
\end{center}
\vspace{1cm}

\section{Research Design}
\section{Data Source: NHANES (2017-2018)}
\section{Data Pre-processing and Feature Engineering}
\section{Biological Age Calculation (PhenoAge)}
\section{Model Development}
\section{Evaluation Methods and Metrics}

% ========================================
% CHAPTER 4: EXPLORATORY DATA ANALYSIS
% ========================================
\newpage
\chapter{Exploratory Data Analysis}

\begin{center}
\fbox{
\begin{minipage}{0.8\textwidth}
\textbf{4.1 Introduction}

\textbf{4.2 Data Overview and Quality Assessment}

\textbf{4.3 Demographic Characteristics}

\textbf{4.4 Biomarker Distributions and Outlier Analysis}

\textbf{4.5 Correlation Analysis}

\textbf{4.6 Age Acceleration Distribution}

\textbf{4.7 Insights from EDA}
\end{minipage}
}
\end{center}
\vspace{1cm}

\section{Introduction}
\section{Data Overview and Quality Assessment}
\section{Demographic Characteristics}
\section{Biomarker Distributions and Outlier Analysis}
\section{Correlation Analysis}
\section{Age Acceleration Distribution}
\section{Insights from EDA}

% ========================================
% CHAPTER 5: MODEL IMPLEMENTATION
% ========================================
\newpage
\chapter{Model Implementation and Evaluation}

\begin{center}
\fbox{
\begin{minipage}{0.8\textwidth}
\textbf{5.1 Introduction}

\textbf{5.2 Model Training and Implementation}

\textbf{5.3 Model Evaluation Results}
\end{minipage}
}
\end{center}
\vspace{1cm}

\section{Introduction}
\section{Model Training and Implementation}
\section{Model Evaluation Results}

% ========================================
% CHAPTER 6: RESULTS AND DISCUSSION
% ========================================
\newpage
\chapter{Results and Discussion}

\begin{center}
\fbox{
\begin{minipage}{0.8\textwidth}
\textbf{6.1 Descriptive Statistics and Cohort Characteristics}

\textbf{6.2 Comparative Model Performance (C-Index)}

\textbf{6.3 Digital Biomarker Importance Analysis}

\textbf{6.4 Actuarial Business Impact: Pricing Simulation}

\textbf{6.5 Policyholder Acceptance}

\textbf{6.6 Summary of Key Results}
\end{minipage}
}
\end{center}
\vspace{1cm}

\section{Descriptive Statistics and Cohort Characteristics}

The final analytical cohort consisted of \textbf{N=4,894} NHANES participants after applying rigorous inclusion criteria.

\section{Comparative Model Performance (C-Index)}
\section{Digital Biomarker Importance Analysis}
\section{Actuarial Business Impact: Pricing Simulation}
\subsection{Gini Coefficient Analysis}
\subsection{Dynamic Premium Pricing}
\section{Policyholder Acceptance}
\section{Summary of Key Results}

% ========================================
% CHAPTER 7: INDUSTRY IMPACT ANALYSIS
% ========================================
\newpage
\chapter{Industry Impact Analysis}

\begin{center}
\fbox{
\begin{minipage}{0.8\textwidth}
\textbf{7.1 Global Implementation Case Studies}

\textbf{7.2 Advantages and Benefits}

\textbf{7.3 Challenges and Modifications}

\textbf{7.4 Future Expectations and Industry Trends}

\textbf{7.5 Impact on the Egyptian Insurance Market}

\textbf{7.6 Economic Impact Quantification}
\end{minipage}
}
\end{center}
\vspace{1cm}

\section{Global Implementation Case Studies}
\section{Advantages and Benefits}
\section{Challenges and Modifications}
\section{Future Expectations and Industry Trends}
\section{Impact on the Egyptian Insurance Market}
\section{Economic Impact Quantification}

% ========================================
% CHAPTER 8: CONCLUSION AND RECOMMENDATIONS
% ========================================
\newpage
\chapter{Conclusion and Recommendations}

\begin{center}
\fbox{
\begin{minipage}{0.8\textwidth}
\textbf{8.1 Summary of Contributions}

\textbf{8.2 Implications for the Insurance Industry}

\textbf{8.3 Limitations}

\textbf{8.4 Future Research Directions}

\textbf{8.5 Final Remarks}
\end{minipage}
}
\end{center}
\vspace{1cm}

\section{Summary of Contributions}
\section{Implications for the Insurance Industry}
\section{Limitations}
\section{Future Research Directions}
\section{Final Remarks}

% ========================================
% REFERENCES
% ========================================
\newpage
\chapter*{References}
\addcontentsline{toc}{chapter}{References}

\begin{enumerate}
\item Abraham, M. (2016). Wearable technology: A health-and-care actuary's perspective. The Actuary.
\item Accenture. (2019). Global Financial Services Consumer Study: Insurance. Accenture Research, pp. 1-24. https://www.accenture.com/us-en/insights/financial-services/global-financial-services-consumer-study
\item AIA Vitality Australia. (2021). Member Engagement and Retention Report. AIA Group Research, Sydney.
\item Banaee, H., Ahmed, M. U., \& Loutfi, A. (2013). Data mining for wearable sensors in health monitoring systems: A review of recent trends and challenges. Sensors, 13(12), 17472-17500.
\item Brooks, B., Hershfield, H. E., \& Shu, S. B. (2020). The future self in insurance and retirement savings decisions. Journal of Risk and Insurance, 87(4), 917-943.
\item Chen, T., \& Guestrin, C. (2016). XGBoost: A scalable tree boosting system. Proceedings of the 22nd ACM SIGKDD International Conference on Knowledge Discovery and Data Mining, 785-794.
\item Chen, Y., Qiu, W., Ou, R., \& Huang, C. (2020). A contract-based insurance incentive mechanism boosted by wearable technology. IEEE Internet of Things Journal.
\item CIPFA. (2015). Prevention: Better Than the Cure: Public Health and the Future of Healthcare Funding. CIPFA.org.
\item Cisco. (2018). Cisco Edge-to-Enterprise IoT Analytics for Electric Utilities Solution Overview.
\item Cox, D. R. (1972). Regression models and life-tables. Journal of the Royal Statistical Society: Series B (Methodological), 34(2), 187-202.
\item Culnan, M. J., \& Armstrong, P. K. (1999). Information privacy concerns, procedural fairness, and impersonal trust. Organization Science, 10(1), 104-115.
\item de Zambotti, M., Rosas, L., Colrain, I. M., \& Baker, F. C. (2017). The Sleep of the Ring: Comparison of the Oura Sleep Tracker Against Polysomnography. Behavioral Sleep Medicine, 1-15.
\item Deloitte. (2024). Insurance outlook 2024: Navigating transformation through technology and innovation. Deloitte Center for Financial Services.
\item Dinev, T., \& Hart, P. (2006). An extended privacy calculus model for e-commerce transactions. Information Systems Research, 17(1), 61-80.
\item Discovery Health. (2020). Vitality Digital Innovation Report. Discovery Limited, Johannesburg.
\item Egyptian Financial Regulatory Authority (FRA). (2023). Insurance Market Report 2023. FRA Publications, Cairo, Egypt.
\item EIOPA. (2024). Guidelines on the Use of Artificial Intelligence in Insurance. EIOPA Publications.
\item Erdaş, Ç. B., \& Güney, S. (2021). Human Activity Recognition by Using Different Deep Learning Approaches for Wearable Sensors.
\item European Union. (2018). General Data Protection Regulation (GDPR). Regulation (EU) 2016/679.
\item FinTech Global. (2019). Global InsurTech Funding Tops \$3bn in 2018.
\item Gen Re. (2021). Wearables and Health Insurance: A German Consumer Study. Gen Reinsurance Research, Cologne.
\item Generali. (2023). Annual Report: Vitality Program Performance. Generali Group, Trieste.
\item GlobalData. (2022). Consumer Insurance Survey: Attitudes Towards Wearable Technology in Insurance. GlobalData Financial Services, London.
\item GSMA. (2024). Mobile Economy Middle East and North Africa 2024. GSMA Intelligence, London.
\item Hafner, M., Pollard, J., \& Van Stolk, C. (2018). Incentives and Physical Activity: An Assessment of the Association Between Vitality’s Active Rewards and Apple Watch Benefit. Rand Corporation.
\item Henckaerts, R., Côte, M. P., Antonio, K., \& Verbelen, R. (2018). Boosting insights in insurance tariff plans with tree-based machine learning methods. North American Actuarial Journal, 22(2), 255-285.
\item Horvath, S. (2013). DNA methylation age of human tissues and cell types. Genome Biology, 14(10), R115.
\item International Diabetes Federation (IDF). (2023). IDF Diabetes Atlas 10th Edition. IDF Publications, Brussels.
\item John Hancock. (2018). John Hancock Adds Interactive Element to All New Life Insurance Policies. Press Release, Boston, MA.
\item Kahneman, D., \& Tversky, A. (1979). Prospect theory: An analysis of decision under risk. Econometrica, 47(2), 263-291.
\item Katzman, J. L., Shaham, U., Cloninger, A., Bates, J., Jiang, T., \& Kluger, Y. (2018). DeepSurv: Personalized treatment recommender system using a Cox proportional hazards deep neural network. BMC Medical Research Methodology, 18(1), 24.
\item Levine, M. E., Lu, A. T., Quach, A., Chen, B. H., Assimes, T. L., Bandinelli, S., ... \& Horvath, S. (2018). An epigenetic biomarker of aging for lifespan and healthspan. Aging (Albany NY), 10(4), 573.
\item Li, X., Dunn, J., Salins, D., et al. (2017). Digital Health: Tracking Physiomes and Activity Using Wearable Biosensors Reveals Useful Health-Related Information. PLoS Biology.
\item LIMRA/LOMA. (2024). 2024 Insurance Barometer Study: Consumer Attitudes on Wellness Programs. LIMRA Research, Hartford, CT.
\item Lundberg, S. M., \& Lee, S. I. (2017). A unified approach to interpreting model predictions (SHAP). Advances in Neural Information Processing Systems, 30.
\item Majumder, S., Mondal, T., \& Deen, M. J. (2017). Wearable sensors for remote health monitoring. Sensors, 17(1), 130.
\item McCrea, M., \& Farrell, M. (2018). A conceptual model for pricing health and life insurance using wearable technology. Risk Management and Insurance Review, 21(3), 389-411.
\item McKinsey \& Company. (2023). The future of insurance: How artificial intelligence is transforming the industry. McKinsey Global Institute.
\item Missov, T., Németh, L., \& Dańko, M. (2016). How much can we trust life tables? Sensitivity of mortality measures to right-censoring treatment. Palgrave Communications, 2, 15049.
\item NAIC. (2023). Model Bulletin on the Use of Artificial Intelligence in Insurance. NAIC Publications.
\item National Institute for Health and Care Excellence (NICE). (2014). Behavior Change: Individual Changes. Public Health Guideline.
\item Nienaber, A. M., Hofeditz, M., \& Searle, R. (2023). Trust and willingness to share personal health data with insurers. Journal of Risk and Insurance, 90(2), 389-418.
\item O'Neil, C. (2016). Weapons of math destruction: How big data increases inequality and threatens democracy. Crown.
\item Park, S., Choi, J., Lee, S., et al. (2021). Determinants of consumers' adoption of wearable-based health insurance. JMIR mHealth and uHealth, 9(9), e14074. https://doi.org/10.2196/14074
\item Pyrkov, T. V., Slipensky, K., Barg, M., et al. (2021). Extracting biological age from biomedical data via deep learning: Too much of a good thing? Scientific Reports, 11, 5210. https://doi.org/10.1038/s41598-021-84345-3
\item Richman, R. (2021). Machine learning with applications in actuarial science. North American Actuarial Journal, 25(sup1), S315-S321.
\item Schrack, J. A., Cooper, R., Al-Ghatrif, M., ... \& NHANES Consortium. (2018). Calibrating the NHANES wrist-worn accelerometer to estimate physical activity in older adults. Journal of Gerontology: Series A, 73(10).
\item Shim, J., Kim, H., Youn, J., et al. (2023). Wearable-based accelerometer activity profile as digital biomarker of inflammation, biological age, and mortality using hierarchical clustering analysis in NHANES 2011-2014. Nature Communications, 14, 7832. https://doi.org/10.1038/s41467-023-43681-6
\item Spender, A., Bullen, C., Altmann-Richer, L., et al. (2018). Wearables and the internet of things: considerations for the life and health insurance industry. British Actuarial Journal, 24, e22.
\item Statista. (2018). Impact of health insurance on the use of connected health devices in Japan.
\item Statista. (2019). Willingness to use insurance technologies for cheaper premium by technology U.S.
\item Statista. (2021). Clinician's opinions on wearables use lowering health premiums by 2031.
\item Swiss Re. (2024). Global Insurance Market Outlook 2024. Swiss Re Institute, Zurich.
\item Thaler, R. H., \& Sunstein, C. R. (2008). Nudge: Improving decisions about health, wealth, and happiness. Yale University Press.
\item ValuePenguin. (2022). Fitness Trackers and Health Insurance Discounts Survey. LendingTree Research, Charlotte, NC.
\item Vaupel, J. W., Manton, K. G., \& Stallard, E. (1979). The impact of heterogeneity in individual frailty on the dynamics of mortality. Demography, 16(3), 439-454.
\item Vitality Group \& London School of Economics. (2023). Seven-Year Impact Study: Wearables and Mortality Outcomes. LSE Health Working Paper.
\item Volpp, K. G., Asch, D. A., Galvin, R., & Loewenstein, G. (2011). Redesigning Employee Health Incentives—Lessons from Behavioral Economics. New England Journal of Medicine, 365, 388-390.
\item Wüthrich, M. V., \& Merz, M. (2023). Statistical foundations of actuarial learning and its applications. Springer Actuarial.
\item Xu, H., Luo, X., Carroll, J. M., \& Rosson, M. B. (2011). The personalization privacy paradox. Information Technology \& People, 24(4), 315-334.
\end{enumerate}

% ========================================
% APPENDIX
% ========================================
\newpage
\appendix
\chapter{Appendix}

\section{Python Environment Setup}
\section{PhenoAge Calculation Code}
\section{DeepSurv Model Architecture}
\section{Movement Fragmentation Calculation}

\end{document}
